%%%%%%%%%%%%%%%%%%%%%%%%%%%%%%%%%%%%%%%%%
% Friggeri Resume/CV
% XeLaTeX Template
% Version 1.2 (3/5/15)
%
% This template has been downloaded from:
% http://www.LaTeXTemplates.com
%
% Original author:
% Adrien Friggeri (adrien@friggeri.net)
% https://github.com/afriggeri/CV
%
% License:
% CC BY-NC-SA 3.0 (http://creativecommons.org/licenses/by-nc-sa/3.0/)
%
% Important notes:
% This template needs to be compiled with XeLaTeX and the bibliography, if used,
% needs to be compiled with biber rather than bibtex.
%
%%%%%%%%%%%%%%%%%%%%%%%%%%%%%%%%%%%%%%%%%

\documentclass[]{friggeri-cv} % Add 'print' as an option into the square bracket to remove colors from this template for printing

%\addbibresource{publications.bib} % Specify the bibliography file to include publications

\begin{document}

\header{Sharad }{Shriram}{Computer Science Graduate} % Your name and current job title/field

%----------------------------------------------------------------------------------------
%	SIDEBAR SECTION
%----------------------------------------------------------------------------------------

\begin{aside} % In the aside, each new line forces a line break
\section{Contact}
New No: 12 Old No:100
Thirumalaisamy Street
Coimbatore 641044
Tamil Nadu
India
~
%+0 (000) 111 1111 
+91 99420 88026
~
\href{mailto:sharad.sriram@gmail.com}{sharad.sriram@gmail.com}
\href{https://ssharad.github.io}{https://ssharad.github.io}
\href{https://github.com/ssharad}{Github://ssharad}
\href{https://in.linkedin.com/in/sharad10694}{LnkdIn://Sharad Shriram}
\href{https://www.quora.com/profile/Sharad-Shriram}{Quora://Sharad Shriram}
\section{Programming}
{\color{red} $\varheartsuit$} Python
Django, C++, PHP,
MySQL, Latex, CSS3,
Javascript \& HTML5
\section{Platforms}
Arduino, MSP430, CC3200,
Beaglebone, RasberryPi
\section{Software}
Energia, Jekyll, Github, Linux
\section{Languages}
Tamil 
English 
Hindi 
%----------------------------------------------------------------------------------------
%	AWARDS SECTION
%----------------------------------------------------------------------------------------
\section{Awards}
\textbf{Best Paper - Student} at the 2014 International Conference on Intelligent Computing Applications for my paper "Increasing the Internet Speed and Bandwidth Using the Laws of Physics".
%------------------------------------------------
\section{Interests}
\textbf{professional:} research, IoT, data analysis, web design, web app creation, software design\\ \textbf{personal:} writing, music, cooking, basketball, running
\end{aside}

%----------------------------------------------------------------------------------------
%   ABOUT ME SECTION (similar to the interests
%----------------------------------------------------------------------------------------
\section{About Me!}
I enjoy building software and web applications. An aspiring researcher with interests in Internet of Things, Software Engineering, Big Data, Algorithms and Machine Learning. I'm looking for a position in the capacity of a <role> with you. In all, I'm a result driven preson, eager to learn and feel comfortable in a fast-changing environment.

%----------------------------------------------------------------------------------------
%	EDUCATION SECTION
%----------------------------------------------------------------------------------------

\section{Education}

\begin{entrylist}

%------------------------------------------------

%\entry
%{2011--2012}
%{Masters {\normalfont of Commerce}}
%{The University of California, Berkeley}
%{\emph{Money Is The Root Of All Evil -- Or Is It?} \\ This thesis explored the idea that money has been %the cause of untold anguish and suffering in the world. I found that it has, in fact, not.}

%------------------------------------------------

\entry
{2012--2016}
{Bachelor {\normalfont of Technology}}
{Amrita Vishwa Vidyapeetham (University)}
{Specialization in Computer Science and Engineering}

%------------------------------------------------

\end{entrylist}

%----------------------------------------------------------------------------------------
%	WORK EXPERIENCE SECTION
%----------------------------------------------------------------------------------------

\section{Experience}

%\subsection{Full Time}

%\begin{entrylist}

%------------------------------------------------

%\entry
%{2016--Now}
%{LEHMAN BROTHERS}
%{Los Angeles, California}
%{\emph{1\textsuperscript{st} Year Analyst} \\
%Developed spreadsheets for risk analysis on exotic derivatives on a wide array of commodities (ags, oils, precious and base metals), managed blotter and secondary trades on structured notes, liaised with Middle Office, Sales and Structuring for bookkeeping. \\
%Detailed achievements:
%\begin{itemize}
%\item Learned how to make amazing coffee
%\item Finally determined the reason for \textsc{PC LOAD LETTER}:
%\end{itemize}}

%------------------------------------------------

%\end{entrylist}

%\subsection{Part Time}

\begin{entrylist}

%\entry
%{Dec'15--Aug'16}
%{Amrita Vishwa Vidyapeetham(University)}
%{Coimbatore, India}
%{\emph{Undergraduate Research Intern} \\
%Worked under Dr.Senthil Kumar.T to develop and implement an algorithm for searching large datasets with Locality Sensitive Hashing Mapreduce \&  as part of a projct funded by IBM, India.}

%------------------------------------------------
\entry
{Sep'14--May'16}
{Mobile and Wireless Networks Lab, Amrita Vishwa Vidyapeetham(University)}
{Coimbatore, India}
{\emph{Undergraduate Research Intern} \\
Worked under Dr.Bagavathi Sivakumar.P to design and build prototypes for our projects on Internet of Things and Machine Learning. We have published two papers from three major projects. }

%------------------------------------------------
\entry
{Jul'15}
{First IEEE International Smart Cities Conference}
{Guadalajara, Mexico}
{\emph{Subreviewer} \\
Reviewed a research paper as part of this conference under Dr.Ryosuke Ando, PC member.}

%------------------------------------------------
\entry
{Jun'15}
{Electronic System Design Lab, Velammal Engineering College}
{Chennai, India}
{\emph{Summer Research Intern} \\
Developed an IoT module as part of the project "Wireless Sensor Network enabled earlier real-time detection of spoilage in stored grains" funded by the Dept. of Science and Technology, Govt. of India.}

%------------------------------------------------
\entry
{May--Aug '14}
{Mobile and Wireless Networks Lab, Amrita Vishwa Vidyapeetham(University)}
{Coimbatore, India}
{\emph{Summer Research Intern} \\
Developed a proof of concept for an attack perspective algorithm to secure communication between wireless personal medical devices.}

%------------------------------------------------
\end{entrylist}

%----------------------------------------------------------------------------------------
%	AWARDS SECTION
%----------------------------------------------------------------------------------------
%\section{Awards}
%\begin{entrylist}
%------------------------------------------------
%\entry
%{2014}
%{Best Paper - Student}
%{International Conference on Intelligent Computing Applications}
%{Awarded to the best student research paper at the conference for the paper "Increasing the Internet Speed and Bandwidth Using the Laws of Physics".}
%------------------------------------------------
%\end{entrylist}

%----------------------------------------------------------------------------------------
%	PUBLICATIONS SECTION
%----------------------------------------------------------------------------------------

\section{Publications}
1. Sharad, S., P. Bagavathi Sivakumar, and V. Anantha Narayanan. "A Novel IoT-Based Energy Management System for Large Scale Data Centers." \textit{In Proceedings of the 2015 ACM Sixth International Conference on Future Energy Systems}, pp. 313-318. ACM, 2015. \\
2. Ananthanarayanan.V , Rajeswari.A , GouthamKashyap. P, Sharad S. "An Attack Perceptive Approach for Reliable and Secure Wireless Connectivity between Medical Devices in Public Environment." \textit{International Journal of Applied Engineering Research}10, no. 3 (2015): 6581-6600.\\
3. Sharad,  S. "Increasing internet speed and bandwidth by using laws of physics." In  \textit{2014 International Conference on Intelligent Computing Applications, ICICA 2014}, pp. 99-103.  Institute of Electrical and Electronics Engineers Inc., 2014.\\
\end{document}